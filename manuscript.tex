%%%%%%%%%%%%%%%%%%%%%%%%%%%%%%%%%%%%%%%%%%%%%%%%%%%%%%%%%%%%
%%% LIVECOMS ARTICLE TEMPLATE FOR BEST PRACTICES GUIDE
%%% ADAPTED FROM ELIFE ARTICLE TEMPLATE (8/10/2017)
%%%%%%%%%%%%%%%%%%%%%%%%%%%%%%%%%%%%%%%%%%%%%%%%%%%%%%%%%%%%
%%% PREAMBLE
\documentclass[9pt,tutorial]{livecoms}
% Use the 'onehalfspacing' option for 1.5 line spacing
% Use the 'doublespacing' option for 2.0 line spacing
% Use the 'lineno' option for adding line numbers.
% The 'bestpractices' option for indicates that this is a best practices guide.
% Omit the bestpractices option to remove the marking as a LiveCoMS paper.
% Please note that these options may affect formatting.

\usepackage{lipsum} % Required to insert dummy text
\usepackage[version=4]{mhchem}
\usepackage{siunitx}
\DeclareSIUnit\Molar{M}
\usepackage[italic]{mathastext}
\graphicspath{{figures/}}


%%%%%%%%%%%%%%%%%%%%%%%%%%%%%%%%%%%%%%%%%%%%%%%%%%%%%%%%%%%%
%%% IMPORTANT USER CONFIGURATION
%%%%%%%%%%%%%%%%%%%%%%%%%%%%%%%%%%%%%%%%%%%%%%%%%%%%%%%%%%%%

\newcommand{\versionnumber}{0.1} % TODO: can we extract this number from a git tag?
\newcommand{\githubrepository}{\url{github.com/markovmodel/pyemma_tutorials}}

%%%%%%%%%%%%%%%%%%%%%%%%%%%%%%%%%%%%%%%%%%%%%%%%%%%%%%%%%%%%
%%% ARTICLE SETUP
%%%%%%%%%%%%%%%%%%%%%%%%%%%%%%%%%%%%%%%%%%%%%%%%%%%%%%%%%%%%
\title{PyEMMA: learning the ropes of Markov state modeling --- v\versionnumber}

\author[1\authfn{1}*]{Christoph Wehmeyer}
\author[1\authfn{1}*]{Martin K. Scherer}
\author[1\authfn{1}*]{Tim Hempel}
\author[1*]{Simon Olsson}
\author[1*]{Frank Noé}
\affil[1]{Department of Mathematics and Computer Science, Freie Universität Berlin, Arnimallee 6, 14195 Berlin, Germany}

\corr{christoph.wehmeyer@fu-berlin.de}{CW}
\corr{m.scherer@fu-berlin.de}{MKS}
\corr{tim.hempel@fu-berlin.de}{TH}
\corr{simon.olsson@fu-berlin.de}{SO}
\corr{frank.noe@fu-berlin.de}{FN}

\contrib[\authfn{1}]{These authors contributed equally to this work}

\blurb{This LiveCoMS document is maintained online on GitHub at \githubrepository{}; to provide feedback, suggestions, or help improve it, please visit the GitHub repository and participate via the issue tracker.}

%%%%%%%%%%%%%%%%%%%%%%%%%%%%%%%%%%%%%%%%%%%%%%%%%%%%%%%%%%%%
%%% ARTICLE START
%%%%%%%%%%%%%%%%%%%%%%%%%%%%%%%%%%%%%%%%%%%%%%%%%%%%%%%%%%%%

\begin{document}

\begin{frontmatter}
\maketitle

\begin{abstract}
This tutorial provides an introduction to the PyEMMA project. Using Jupyter notebooks, we will guide you through the basic functionality as well as the more common advanced mechanisms. Short exercises to self check your learning progress and a notebook on troubleshooting complete this basic introduction.
\end{abstract}

\end{frontmatter}

\section{Introduction}

PyEMMA~\cite{pyemma} (\url{http://emma-project.org}) is a framework for the analysis of molecular dynamics (MD) simulations using Markov state models\cite{schuette-msm,singhal-msm-naming} (MSMs). The package is written in Python (\url{http://python.org}), relies heavily on NumPy/SciPy~\cite{numpy,scipy}, and is compatible with the scikit-learn~\cite{sklearn} framework for machine learning.

\subsection{Scope}

In this tutorial, we assume the reader's familiarity with the basic theory behind the MSM approach (see Sec.~\ref{sec:background}) and focus on usage of PyEMMA. Nevertheless will we mention important theoretical concepts throughout the tutorial.

The tutorial is divided into lessons on specific topics, each accompanied by a Jupyter~\cite{jupyter} notebook containing code, instructions, and exercises. The lessons start with a showcase of the PyEMMA workflow and follow up with in-depth lessons on specific topics.

\section{Prerequisites}

In the following, we summarize the recommended theoretical background knowlegde of Markov state modeling for this tutorial as well as the software requirements to work through the lessons.

\subsection{Background knowledge}
\label{sec:background}

For those unfamiliar with Markov state modeling, ``\emph{Markov State Models: From an Art to a Science}''~\cite{msm-brooke} provides a comprehensive overview, while ``\emph{Markov models of molecular kinetics: Generation and validation}''~\cite{msm-jhp} describes the mathematical theory. 

In addition to the literature on theory and application of Markov state modeling~\cite{msm-book,buchete-msm-2008,noe-tmat-sampling,bowman-msm-2009,noe-folding-pathways,sarich-msm-quality,noe-fingerprints,noe-dy-neut-scatt,Chodera2014,ben-rev-msm,simon-mech-mod-nmr,oom-feliks}, we also recommend the literature on time-lagged independent component analysis (TICA)~\cite{tica,tica3,tica2,kinetic-maps}, transition path theory (TPT)~\cite{weinan-tpt,metzner-msm-tpt}, hidden Markov state models (HMMs)~\cite{noe-proj-hid-msm,hmm-baum-welch-alg,hmm-tutorial}, and variational techniques~\cite{noe-vac,vamp-preprint,gmrq}, as these topics play important roles within the standard MSM workflow.

\subsection{Software/system requirements}

We utilize Jupyter~\cite{jupyter} notebooks to show code examples along with figures and interactive widgets to display molecules. The user can install all needed packages with only one command if they use the \texttt{conda} command provided by the Anaconda Python stack (\url{https://anaconda.com}). We rely on Anaconda, because it resolves and installs dependencies, and provides pre-compiled versions of hard to build packages like SciPy.

The installation will contain a launcher command to start the Jupyter notebook server as well as the notebook files. The data for the demonstrated test systems is downloaded upon the first use and is cached for future invocations of the tutorial.

The underlying software stack for running the tutorial importantly consists of:
\begin{itemize}
\item NumPy -- number crunching~\cite{numpy}
\item SciPy -- linear algebra, sparse computations~\cite{scipy}
\item Matplotlib -- Visualization~\cite{matplotlib}
\item \textbf{PyEMMA} -- MSM/HMM estimation, validation, analysis, and visualization~\cite{pyemma}
\item mdtraj -- reading MD simulation data, calculate observables from MD~\cite{mdtraj}
\end{itemize}

The software is currently supported for Python versions $3.5$ and $3.6$ on the operating systems Linux, OSX, and Windows.

If one does not want to use Anaconda, they can also perform a manual installation via the pip installer or use the Binder service to view and run the tutorials online in any browser.

\section{Content and links}

This tutorial consists of nine Jupyter notebooks which introduce the basic features of PyEMMA. The first notebook (00), which we will summarize in the following, showcases the entire estimation, validation, and analysis workflow. The eight subsequent notebooks (01--08) provide in-depth lessons on specific topics from the workflow, an introduction to HMMs, and guidelines how to deal with common problems during an MSM estimation.

\subsection{The PyEMMA workflow}

In short, the workflow consists of extracting molecular features from the raw data (01) which we transform into a suitable, low-dimensional subspace and discretize (02). Then, we estimate a maximum likelihood or Bayesian MSM from the discrete trajetcories and perform validation tests (03). We can then analize the stationary and kinetic properties of the MSM (04), apply transition path theory (TPT) to identify the pathways of conformation change (05), and compute expectation values for experimental observables (06).

In the showcase notebook, we analyze the backbone conformation dynamics of a pentapeptide (Trp-Leu-Ala-Leu-Leu, Fig.~\ref{fig:io-to-ck}a) using 25 independently simulated trajectories with implicit solvent and a trajectory saving interval of $0.1$ ns.~\cite{pyemma}

\subsection{Estimation and validation}

\begin{figure}
\includegraphics{figure_1}
\caption{Exemplary analysis of a pentapeptide's backbone conformational dynamics: (a) The studied pentapeptide. (b) The VAMP2 score indicates which of the tested featurization contains the most kinetic variance. (c) The first two independent components of a TICA projection at lag time $\tau=0.5$ ns show rare transition events. (d) Projection of the sample distribution and 200 $k$-means cluster centers onto the first two independent TICA components. (e) The convergence behavior of the first four implied timescales indicates that a lag time of $\tau=0.5$ ns is suitable for estimating an MSM; the shaded areas indicate $95\%$ confidence intervals. (f) A Chapman-Kolmogorov test shows that an MSM estimated at lag time $\tau=0.5$ ns under the assumption of four metastable states accurately predicts the kinetic behavior on longer timescales; the shaded areas indicate $95\%$ confidence intervals.}
\label{fig:io-to-ck}
\end{figure}

As it is unknown which molecular features are best at representing the slow dynamics of the pentapeptide, we start with a broad systematic analysis: we compute the (cross-validated) VAMP-2~\cite{vamp-preprint} score at lag time $0.5$ ns for three different featurizations and find that backbone torsions contain more kinetic variance than the backbone's heavy atoms' positions or distances between them (Fig.~\ref{fig:io-to-ck}b).

A subsequent TICA~\cite{tica,kinetic-maps} transformation of the backbone torsions at lag time $0.5$ ns yields a four dimensional subspace. The sample density projected onto the first two independent components (Fig.~\ref{fig:io-to-ck}c) exhibits several maxima and a time series visualization of said components from the first trajectory (Fig.~\ref{fig:io-to-ck}d) shows discrete jumps between these density maxima. We thus assume that our transformed feature selection describes one or more metastable processes.

We then proceed with a $k$-means-based discretization of the four dimensional TICA subspace using $k=75$ cluster centers and perform the first validation test, i.e., the estimation of implied timescales (ITS) $t_i$, $i=0, 1,\dots$.~\cite{swope-its} The ITS $t_i$ approximates the decorrelation time of the $i^\textrm{th}$ process and is computed from the eigenvalues $\lambda_i$ of the MSM's transition matrix via
\begin{equation}
\label{eq:its}
t_i = \frac{-\tau}{\ln\left|\lambda_i(\tau)\right|}
\end{equation}
with $\tau$ being the lag time. In this validation test, we look for ITS convergence and choose $\tau$ accordingly, i.e., within a range where the ITS are approximately invariant; the uncertainty of the implied timescales is quantified based upon Markov models sampled according to a Bayesian scheme.~\cite{noe-tmat-sampling} In our example, we find the three slowest ITS to converge quickly and to be constant within a $95\%$ confidence interval for lag times above $0.5$ ns (Fig.~\ref{fig:io-to-ck}e). We can further see a converged fourth process (cyan line) which, however, decays for lag times above $2.5$ ns and might not be well resolved in the final model.

Finally, we estimate a (Bayesian) MSM at a the lag time $\tau=0.5$ ns and validate the model using a Chapman-Kolmogorov (CK) test which compares the right and the left side of the Chapman-Kolmogorov equation
\begin{equation}
\label{eq:ck}
T(k \tau) = T^k(\tau)
\end{equation}
with $T$ being the MSM's transition matrix. PyEMMA automatically estimates a new MSM transition matrix at lag time $k \tau$ and propagates the original transition matrix by the $k^\textrm{th}$ power. As we have resolved three slow timescales in the ITS test, we assume to have four metastable states present in the data and the subsequent CK test shows that our model accurately indeed predicts the kinetic behavior on longer timescales accurately (Fig.~\ref{fig:io-to-ck}f).

\subsection{Analyzing the MSM}

\begin{figure}
\includegraphics{figure_2}
\caption{Exemplary analysis of a pentapeptide's backbone conformational dynamics: (a) The reweighted free energy surface in the projection of the first two independent components exhibits five minima which (b) PCCA++ identifies as four metastable states. (c) The second right eigenvector shows that the slowest process shifts probability between the least probable state (0) and the two most probable states (2, 3), whereas (d) the commitor $1\to2$ indicates that the most probable state 3 acts as a transition state for transitions between states 1 and 2.}
\label{fig:msm-analysis}
\end{figure}

We can now directly extract several thermodynamic and kinetic properties from the estimated and validated model. An example of the former is the free energy surface in the projection of the first two TICA components (Fig.~\ref{fig:msm-analysis}a) reweighted based on the MSM's stationary distribution.

A fuzzy spectral clustering using the PCCA++ algorithm~\cite{pcca++} allows us to coarse grain the $k$-means microstates into four metastable macrostates (Fig.~\ref{fig:msm-analysis}b) $\mathcal{S}_i$, $i=0,\dots,3$, for which we then approximate the absolute probabilities and free energies
\[ \begin{array}{ccc}
\textrm{macrostate } i & \pi_{\mathcal{S}_i} & G_i / \textrm{k}_\textrm{B} T \\
\hline
0 & 0.004 & 5.565 \\
1 & 0.013 & 4.314 \\
2 & 0.021 & 3.864 \\
3 & 0.962 & 0.039
\end{array}\]
the latter via
\begin{equation}
\label{eq:fe}
G_i = - \textrm{k}_\textrm{B} T \ln\left(\sum\limits_{j\in \mathcal{S}_i} \pi_j\right),
\end{equation}
where $\pi_j$ denotes the MSM's stationary weight of the $j^\textrm{th}$ microstate.

To understand which slow process occur, we refer to the the MSM's (right) eigenvectors as they contain information about what configurational changes are happening on which timescales. The first right eigenvector corresponds to the stationary process and is constant $1$. The second right eigenvector, however, corresponds the the slowest process and its minimal and maximal components indicate the microstates between the process shifts probability. The relaxation timescale of this exchange process is exactly the implied timescale. In the usual projection, we identify the slowest process as a probability shift between macrostate $0$ and the rest of the system, with macrostates $2$ and $3$ in particular (Fig.~\ref{fig:msm-analysis}c).

The mean first passage times (MFPTs) for the slowest process compute to
\[ \begin{array}{crcr}
\textrm{direction} & \textrm{mean / ns} && \textrm{std / ns} \\
\hline
\mathcal{S}_0 \to \mathcal{S}_{(1,2,3)} & 8.9 & \pm & 1.8 \\
\mathcal{S}_{(1,2,3)} \to \mathcal{S}_0 & 2487.5 & \pm &  504.7
\end{array}\]
using the Bayesian MSM.

As an example for TPT, we compute the flux between macrostates $1$ and $2$ (Fig.~\ref{fig:msm-analysis}d). The committor projection onto the first two TICA components shows that it is constant within the metastable states defined above. Transition regions (macrostate $3$) can be identified by committor values $\approx \frac{1}{2}$.

Next we compute the Trp-1 solvent accessible surface area (SASA) using the mdtraj~\cite{mdtraj} package. The SASA is a reasonable approximation for the instantaneous tryptophan flourescence signal and the auto correlation function (ACF) of this signal can be measured in spectroscopy experiments. In our usual projection, we observe a high variance of the SASA in all macrostates except $0$ (Fig.~\ref{fig:msm-analysis}e), indicating that Trp-1 has only a low solvent exposure in this macrostate. The ACF itself yields only a weak signal (Fig.~\ref{fig:msm-analysis}f, top panel). If, however, we prepare the system in macrostate $0$ (non-equilibrium condition), the relaxation towards equilibrium yields a much stronger signal (Fig.~\ref{fig:msm-analysis}f, bottom panel).

Finally, we sample $100$ super-imposed molecular conformations from each of the $4$ macrostates (Fig.~\ref{fig:msm-analysis}g). We observe that, in macrostates $0$ and $1$, the backbone appears to shield the Trp-1 (shown in red) from the solvent. In the more probable macrostates $2$ and $3$, the Trp-1 appears to be much more exposed.

\subsection{Summary}

We have summarized in this section how to conduct an MSM-based analysis of molecular dynamcis data using PyEMMA. For the full analysis, please refer to the first notebook (00). All notebooks as well as detailed installation instructions are available on \githubrepository{}.

\section{Author Contributions}
%%%%%%%%%%%%%%%%
% This section mustt describe the actual contributions of
% author. Since this is an electronic-only journal, there is
% no length limit when you describe the authors' contributions,
% so we recommend describing what they actually did rather than
% simply categorizing them in a small number of
% predefined roles as might be done in other journals.
%
% See the policies ``Policies on Authorship'' section of https://livecoms.github.io
% for more information on deciding on authorship and author order.
%%%%%%%%%%%%%%%%
For a more detailed description of author contributions, see the GitHub issue tracking and changelog at\\\githubrepository{}.

\section{Other Contributions}
%%%%%%%%%%%%%%%
% You should include all people who have filed issues that were
% accepted into the paper, or that upon discussion altered what was in the paper.
% Multiple significant contributions might mean that the contributor
% should be moved to authorship at the discretion of the a
%
% See the policies ``Policies on Authorship'' section of https://livecoms.github.io for
% more information on deciding on authorship and author order.
%%%%%%%%%%%%%%%
We are grateful to Brooke E. Husic for her thorough review of the notebooks in this tutorial as well as Camilla Ventura Santos and the entire computational molecular biology group for valuable discussion and feedback.

For a more detailed description of contributions from the community and others, see the GitHub issue tracking and changelog at \githubrepository{}.

\section{Potentially Conflicting Interests}
%%%%%%%
%Declare any potentially competing interests, financial or otherwise
%%%%%%%

\section{Funding Information}
%%%%%%%
% Authors should acknowledge funding sources here. Reference specific grants.
%%%%%%%
TH acknowledges financial support by SFB/TRR 186. SO acknowledges a postdoctral fellowship from the Alexander von Humboldt Stiftung.

\bibliography{literature}

%%%%%%%%%%%%%%%%%%%%%%%%%%%%%%%%%%%%%%%%%%%%%%%%%%%%%%%%%%%%
%%% APPENDICES
%%%%%%%%%%%%%%%%%%%%%%%%%%%%%%%%%%%%%%%%%%%%%%%%%%%%%%%%%%%%

%\appendix


\end{document}
